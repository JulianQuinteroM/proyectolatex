\documentclass{article}
\usepackage[utf8]{inputenc}
\usepackage[spanish]{babel}
\usepackage{listings}
\usepackage{graphicx}
\graphicspath{ {images/} }
\usepackage{cite}

\begin{document}

\begin{titlepage}
    \begin{center}
        \vspace*{1cm}
            
        \Huge
        \textbf{Calistenia}
            
        \vspace{0.5cm}
        \LARGE
        Subtítulo
            
        \vspace{1.5cm}
            
        \textbf{Nombres y Apellidos del autor}
        \newline Julián David Quintero Marín
        \newline
        \newline
        Informática II
        \vfill
            
        \vspace{0.8cm}
            
        \Large
        Despartamento de Ingeniería Electrónica y Telecomunicaciones\\
        Universidad de Antioquia\\
        Medellín\\
        Marzo de 2021
            
    \end{center}
\end{titlepage}

\tableofcontents
\newpage
\section{Sección introductoria}\label{intro}
En este trabajo se busca la forma de dar instrucciones a tres personas a través de indicaciones específicas para que realicen una acción determinada, sin explicar ningun paso, sólamente experimentando qué pueden entender sólo con las indicaciones entregadas.

\section{Sección de contenido} \label{contenido}
1. Levantar la hoja que está sobre las tarjetas fragilmente para que no se arrugue y moverla a otra posición, asugurándose de que la hoja quede sobre una superficie horizontal y plana.
\newline
2. Coger las tarjetas usando los dedos pulgar e índice y colocandolos cada uno en las puntas rojas de las tarjetas, luego levantarlas de forma que el punto verde de las tarjetas quede en la parte inferior.
\newline
3. Manteniendo la posición de las cartas, ponerlas sobre la línea (Y) dibujada en la hoja, de forma que la línea quede paralela a la base de las tarjetas.
\newline
4. Usar el delo del corazón y el anular para separar las tarjetas por su parte inferior, pero manteniendo el pulgar y el índice uniendo la parte superior de las tarjetas. Separar cada carta hasta la línea (X) y (Z), pero manteniendo la parte superior donde se unen estas, sobre la línea (Y).
\newline
5. Al formar una pirámide, asegurarse de que las cartas tengan un equilibrio para mantenerse y luego soltarlas.

En la sección \ref{imagenes}, se presentará como añadir ilustraciones al texto.

\section{Inclusión de imágenes} \label{imagenes}

En la Figura (\ref{fig:Cartas}), se presenta las tarjetas que se utilizaron en la actividad.

\begin{figure}[h]
\includegraphics[width=4cm]{Cartas.jpg}
\centering
\label{fig:Cartas}
\end{figure}

En la Figura (\ref{fig:Hoja_Base}), se presenta la imágen de la hoja base de las tarjetas

\begin{figure}[h]
\includegraphics[width=4cm]{Hoja_Base.jpg}
\centering
\label{fig:Hoja_Base}
\end{figure}


\bibliographystyle{IEEEtran}
\bibliography{references}

\end{document}
